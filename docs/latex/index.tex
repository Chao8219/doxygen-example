This file is an example of documentation that is generated by Doxygen.

You can find the Doxygen info \href{http://www.stack.nl/~dimitri/doxygen/index.html}{\tt here}.

You can find the license info \href{https://choosealicense.com/}{\tt here}.

I made a logo from wix.\+com\+:



Looks awesome right?

Also, see \href{https://chao8219.github.io/doxygen-example/}{\tt here} to enter github pages.

\section*{Contents}


\begin{DoxyEnumerate}
\item \href{#doxygen-example}{\tt Introduction}
\item \href{#doxygen-note}{\tt Doxygen Note}
\item \href{#to-do-list}{\tt To-\/do List}
\end{DoxyEnumerate}

\section*{Doxygen Note}

\subsection*{Comment Blocks}


\begin{DoxyEnumerate}
\item You can use either ``` /$\ast$$\ast$ Line 1
\begin{DoxyItemize}
\item Line 2
\item Line 3 $\ast$/ Constructor() ``` or
\end{DoxyItemize}

``` Constructor() /$\ast$$\ast$$<$ Line 1
\begin{DoxyItemize}
\item Line 2
\item Line 3 $\ast$/ ``` \href{#contents}{\tt back to top}
\end{DoxyItemize}
\end{DoxyEnumerate}

\subsection*{Special Comannds}


\begin{DoxyEnumerate}
\item To jump to a function or parameter, you can use {\ttfamily @see F\+OO} to point it out.
\item To display a parameter and describe its usage, you can use {\ttfamily @param F\+OO The example parameter.}
\item To start a new line, one could use {\ttfamily $<$br$>$} or {\ttfamily \textbackslash{}n}.
\end{DoxyEnumerate}

\href{#contents}{\tt back to top}

\section*{To-\/do List}


\begin{DoxyItemize}
\item \mbox{[}x\mbox{]} Make table of contents.
\item \mbox{[}x\mbox{]} Add more comments in this example.
\end{DoxyItemize}

\href{#contents}{\tt back to top} 